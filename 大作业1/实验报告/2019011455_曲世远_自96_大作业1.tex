%不需修改部分
\documentclass[UTF8]{ctexart}  %使用中文版的article文档类型排版,并选择UTF8编码格式
\usepackage{amsmath}
\usepackage[colorlinks,linkcolor=black,anchorcolor=blue,citecolor=green]{hyperref}    %修改特定内容的颜色
\usepackage{enumerate}                      %条目包
\usepackage{float}                          %固定表格位置
\usepackage{graphicx}                       %图片包
\usepackage{float}                          %设置图片浮动位置的宏包
\usepackage{subfigure}                      %插入多图时用子图显示的宏包
\usepackage{titlesec}                       %自定义多级标题格式的宏包
\usepackage{geometry}
\usepackage{wrapfig}
\pagestyle{plain}						    %显示页码
\RequirePackage{fix-cm}                     
\usepackage{longtable}
\usepackage{listings}
\usepackage{setspace}
\usepackage{xcolor}
\usepackage{caption}
\usepackage{diagbox}
\usepackage{multirow} %合并多行单元格的宏包
\usepackage{longtable} %不宽但很长的表格可以用longtable宏包来进行分页显示
\usepackage{array} %一般用于数学公式中对数组或矩阵的排版
\usepackage{makecell}% makecell命令对表格单元格中的数据进行一些变换的控制。我们可以使用 \ 命令进行换行,也可以使用p{(宽度)}选项控制列表的宽度
\usepackage{threeparttable} %制作三线表格
\usepackage{booktabs}%s三线表格中的上中下直线线型设置宏包,在这个包中水平线被教程\toprule、midrule、buttomrule。

%各级标题格式自定义
% \titleformat{\section}[block]{\Large\bfseries}{\arabic{section}\quad }{1em}{}[]
% \titleformat{\subsection}[block]{\large\itshape\bfseries}{\arabic{section}.\arabic{subsection}}{1em}{}[]
% \titleformat{\subsubsection}[block]{\normalsize\bfseries}{\arabic{section}.\arabic{subsection}.\arabic{subsubsection}}{1em}{}[]
% \titleformat{\paragraph}[block]{\small\bfseries}{[\arabic{paragraph}]}{1em}{}[]

\linespread{1.5}

\geometry{a4paper,left=2cm,right=2cm,top=2cm,bottom=2cm}

\begin{document}  %开始写文章

\begin{titlepage} %制作封面
    \heiti
    \vspace*{100pt}
    \begin{center}
        \fontsize{30pt}{0} 模\quad 拟\quad 电\quad 子\quad 技\quad 术\quad 基\quad 础\quad \\
        \vspace*{36pt}
        \fontsize{36pt}{0}{综\quad 合\quad 论\quad 文}\\
        \vspace*{48pt}
        \LARGE(2020\ -\ 2021\ 学年度\qquad 春季学期)\\
        \vspace*{48pt}

        \fontsize{20pt}{0} \ \underline{\makebox[450pt]{Title}}\\
        \vspace*{72pt}

        \heiti\Large 姓名\ \ \underline{\makebox[168pt]{\songti Name}}\\
        \heiti\Large 学号\ \ \underline{\makebox[168pt]{\songti ID}}\\
        \heiti\Large 院系\ \ \underline{\makebox[168pt]{\songti 自动化系}}\\
        \heiti\Large 教师\ \ \underline{\makebox[168pt]{\songti Teacher}}\\
        \heiti\Large 时间\ \ \underline{\makebox[168pt]{\songti 2021年6月}}\\
    \end{center}
\end{titlepage}

\newpage

\renewcommand{\abstractname}{\textbf{\Large \heiti 摘要}}

~\\

\begin{abstract}

    ~\\

    \large{    本文xxxxx}

    
    ~\\

    \textbf{关键词}:关键词1、关键词2、关键词3
\end{abstract}


\newpage

\tableofcontents %插入目录
\thispagestyle{empty}

\newpage

\section{引言}

\subsection{其余部分同实验报告模板\footnote{liuzuyan19@mails.tsinghua.edu.cn}}


\newpage

%此处参考文献适用于参考文献较少的情况,需手动设置,若需自动设置可以参考其他方法
\begin{thebibliography}{}
    \bibitem{1}童诗白, 华成英. 模拟电子技术基础(第五版)[M].北京: 高等教育出版社, 2015.
    \bibitem{2}杨祥伦.对数运算电路基本特性的分析和应用实例[J].自动化技术与应用,1983(03):95-101.
    \bibitem{3}模拟电子技术基础课件 第10周
    \bibitem{4}对数运算电路 - 电子常识 - 电子发烧友网\\\url{http://www.elecfans.com/dianzichangshi/20100423216688.html}
    \bibitem{5}Single-Supply, High-Speed, Precision Logarithmic Amplifier datasheet (Rev. A)\\\url{https://www.ti.com.cn/cn/lit/ds/symlink/log114.pdf?ts=1623404510251}
    \bibitem{6}周向明,姚建纯.对数运算电路的温度误差及其补偿[J].电测与仪表,1986(01):11-13.
\end{thebibliography}

\end{document}  %结束写文章